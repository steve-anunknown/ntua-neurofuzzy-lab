\documentclass{article}
\usepackage{fullpage}
\usepackage{amsmath}
\usepackage{graphicx}
\usepackage{listings}
\usepackage{placeins} % for \FloatBarrier
\usepackage[english,greek, main=greek]{babel}
\usepackage[utf8]{inputenc}
\useshorthands{;}
\defineshorthand{;}{?}

\usepackage[explicit]{titlesec} % number after section name
%% number after subsection name
\titleformat{\subsection}
  {\normalfont\large\bfseries}
  {}
  {0em}
  {#1\ \thesubsection}
% avoid numbering the table of contents
\titleformat{\section}
  {\normalfont\Large\bfseries}
  {}
  {0em}
  {\ifnum\value{section}=0\relax #1\else #1\ \thesection\fi}
\newcommand{\eng}[1]{\foreignlanguage{english}{#1}} % shortcut for inserting english into greek text


\title{
    \includegraphics[width=\textwidth]{~/Pictures/emp.png} \\
    \vskip 5cm
    Νευροασαφής Έλεγχος και Εφαρμογές\\
    \large Άσκηση 5η
    \vskip 5cm
}

\author{Αναστάσιος Στέφανος Αναγνώστου\\
        03119051}

\begin{document}

\maketitle
\newpage
\tableofcontents
\newpage

\section{Θέμα}



\section{Θέμα}

Η προσομοίωση \eng{Monte Carlo} για τον υπολογισμό του $\pi$γίνεται παράγοντας
τυχαία ένα πλήθος σημειών με συντεταγμένες $(x, y) \in \{0, 1\}^2$ και
μετρώντας το πλήθος αυτών που βρίσκονται εντός του τεταρτοκύκλιου κύκλου
κέντρου $(0, 0)$ και ακτίνας $1$. Δεδομένου ότι το εμβαδόν του κύκλου είναι
$\pi$ και το εμβαδόν του τετραγώνου είναι $1$, η τιμή του $\pi$ μπορεί να
υπολογιστεί ως

\begin{equation}
    \begin{gathered}
    P(\text{σημείο εντός κύκλου}) = \frac{\text{εμβαδόν τεταρτοκυκλίου}}{\text{εμβαδόν τετραγώνου}} = \frac{\frac{\pi}{4}}{1} \implies \\
    \pi = 4 \cdot P(\text{σημείο εντός κύκλου}) \implies \\
    \pi \approx 4 \cdot \frac{\text{πλήθος σημείων εντός κύκλου}}{\text{πλήθος σημείων}}
    \end{gathered}
\end{equation}

Παρατίθεται κώδικας \eng{Python} που υλοποιεί την προσομοίωση.

\selectlanguage{english}
\begin{lstlisting}[language=Python]
import numpy as np
import random

def monte_carlo(n):
    count = 0
    for _ in range(n):
        x = random.uniform(0, 1)
        y = random.uniform(0, 1)
        if x*x + y*y <= 1:
            count += 1
    return 4 * count / n


if __name__ == '__main__':
    iterations = 10000
    print(monte_carlo(iterations))
\end{lstlisting}
\selectlanguage{greek}

Η αντίστοιχη έξοδος της προσομοίωσης για $10^4$ σημεία είναι $3.1484$ και για
$10^6$ σημεία είναι $3.141075$, πράγματι πολύ κοντά στην τιμή του $\pi$.

\section{Θέμα}

\subsection{Ερώτημα}

Παρατίθεται κώδικας \eng{Python} που υλοποιεί την προσομοίωση. 

\selectlanguage{english}
\begin{lstlisting}[language=Python]
import numpy as np
import random

def simulate_trajectory(initial_state, cost, controller, alpha=0.9, steps=10):
    """
    Simulate a trajectory of the controlled Markov chain.
    
    Parameters:
    - initial_state: The starting state of the system.
    - controller: The controller.
    - alpha: Discount factor for the cost.
    - max_steps: Maximum number of steps to simulate.
    
    Returns:
    - states: List of states visited during the trajectory.
    - cost: Total discounted cost of the trajectory.
    """
    states = [initial_state]
    current_state = initial_state
    total_cost = 0
    for k in range(steps):
        control = controller[current_state-1]
        if control == +1:
            next_state = current_state + 1 if np.random.rand() < 0.5 else current_state
        elif control == -1:
            next_state = current_state - 1 if np.random.rand() < 0.5 else current_state
        else:
            raise ValueError("Control must be +1 or -1")
        
        next_state = max(1, min(10, next_state))  # Ensure the state is within bounds
        
        # Cost function g(x)
        g_x = cost[next_state - 1]
        
        total_cost += (alpha ** k) * g_x
        states.append(next_state)
        current_state = next_state
    
    return states, total_cost
\end{lstlisting}
\selectlanguage{greek}

Η συνάρτηση τρέχει μία προσομοίωση για Κ βήματα και για έναν δεδομένο ελεγκτή.
Σε κάθε βήμα επιχειρεί να εφαρμόσει τον έλεγχο με πιθανότητα $1/2$ να αποτύχει
και να μείνει στην ίδια κατάσταση. Τέλος, επιστρέφει τις καταστάσεις από τις οποίες
διήλθε και το συνολικό κόστος, βάσει ενός \eng{discount factor}.

\subsection{Ερώτημα}

\selectlanguage{english}
\begin{lstlisting}[language=Python]
def q_learning(cost, num_episodes=1000000, alpha=0.1,
                gamma=0.9, epsilon=0.05, max_iter_episode=20):
    """
    Perform Q-learning on the stochastic system.
    
    Parameters:
    - num_episodes: Number of episodes to run.
    - alpha: Learning rate.
    - gamma: Discount factor.
    - epsilon: Exploration rate.
    
    Returns:
    - Q: Learned Q-values.
    """
    Q = np.zeros((10, 2))  # 10 states and 2 actions (+1, -1)
    
    for _ in range(num_episodes):
        state = random.randint(0, 9)  # Start from a random state
        for _ in range(max_iter_episode):
            if random.uniform(0, 1) < epsilon:
                action = random.choice([0, 1])  # Explore
            else:
                action = np.argmin(Q[state])  # Exploit
            flag = random.uniform(0, 1) < 0.5
            next_state = state + 1 if action == 0 and flag else state
            next_state = state - 1 if action == 1 and flag else state
            next_state = max(0, min(9, next_state))
            
            reward = cost[next_state]
            
            best_next_action = np.argmax(Q[next_state])
            td_target = reward + gamma * Q[next_state, best_next_action]
            td_delta = td_target - Q[state, action]
            Q[state, action] += alpha * td_delta
            state = next_state
    return Q
\end{lstlisting}
\selectlanguage{greek}

\end{document}
