\documentclass{article}
\usepackage{graphicx}
\usepackage{listings}
\usepackage{placeins} % for \FloatBarrier
\usepackage[english,greek, main=greek]{babel}
\usepackage[utf8]{inputenc}
\useshorthands{;}
\defineshorthand{;}{?}

\usepackage[explicit]{titlesec} % number after section name
%% number after subsection name
\titleformat{\subsection}
  {\normalfont\large\bfseries}
  {}
  {0em}
  {#1\ \thesubsection}
% avoid numbering the table of contents
\titleformat{\section}
  {\normalfont\Large\bfseries}
  {}
  {0em}
  {\ifnum\value{section}=0\relax #1\else #1\ \thesection\fi}
\newcommand{\eng}[1]{\foreignlanguage{english}{#1}} % shortcut for inserting english into greek text


\title{
    \includegraphics[width=\textwidth]{~/Pictures/emp.png} \\
    \vskip 5cm
    Νευροασαφής Έλεγχος και Εφαρμογές\\
    \large Άσκηση 1η
    \vskip 5cm
}

\author{Αναστάσιος Στέφανος Αναγνώστου\\
        03119051}

\begin{document}

\maketitle
\newpage
\tableofcontents
\newpage

\section{Θέμα}

\subsection{Ερώτημα}

Για τον έλεγχο του συστήματος σχεδιάστηκε ελεγκτής τύπου \eng{Mamdani}. Η
είσοδος του είναι το σφάλμα ταχύτητας και η έξοδος είναι η δύναμη της μηχανής.
Το σφάλμα ταχύτητας καθορίζεται από 3 συναρτήσεις μέλους, όπως φαίνεται παρακάτω:

\selectlanguage{english}
\lstinputlisting[firstline=12, lastline=23]{../ex1.m} 
\selectlanguage{greek}

Χρησιμοποιούνται τραπεζοειδείς συναρτήσεις συμμετοχής

\subsection{Ερώτημα}

\subsection{Ερώτημα}

\section{Θέμα}

\subsection{Ερώτημα}

\subsection{Ερώτημα}

\end{document}
